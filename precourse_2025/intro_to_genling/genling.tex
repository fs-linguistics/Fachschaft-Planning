\documentclass[aspectratio=169,hyperref={unicode}]{beamer}

\usetheme{Szeged}
\usecolortheme{beaver}
\usepackage{fontspec}
\usepackage{xcolor}
\usepackage{ulem}
\usepackage{hyperref}
\usepackage[german]{babel}


\title{Intro to general linguistics}
\author{Erik Zeiner}
\institute{Fachschaft General \& Computational Linguistics\\ \textbf{University of Tübingen}}
\date{WS 2025/26 \\ Pre-course}

\begin{document}
	
	
	\frame{\titlepage}
	
	
	\begin{frame}{Subdisciplines}
		
		\begin{center}
			\includegraphics[scale=0.3]{ling.png}
		\end{center}
		
	\end{frame}
	
	\begin{frame}{Phonetics}
		\begin{columns}
			\begin{column}{0.5\textwidth}
				the study of how humans produce and perceive sounds
				
				\includegraphics[scale=0.2]{speech_anatomy.png}
				\includegraphics[scale=0.25]{spectrogram.jpg}
			\end{column}
			
			\begin{column}{0.5\textwidth}
				\includegraphics[scale=0.25]{ipa.png}
			\end{column}
		\end{columns}
	\end{frame}
	
	\begin{frame}{Phonology}
		the study of how languages systematically organise their sounds / sound systems of languages, \\
		the patterns of how sounds work together within a system
		
		\vfill
		
		\includegraphics[scale=0.4]{english_phonology.png}
		
		\scriptsize{e.g. the consonants found in most dialects of English}
	\end{frame}
	
	\begin{frame}{Morphology}
		the study of words, how they are formed, and how they relate to one another within a language
		
		\includegraphics[scale=0.3]{mor_tree.png} \\
		\scriptsize{e.g. morphology tree of a word}
	\end{frame}
	
	\begin{frame}{Syntax}
		the study of how words combine to form larger units such as phrases and sentences
		
		\includegraphics[scale=0.4]{syntax_tree.jpg}
		
		\scriptsize{e.g. syntax tree of a sentence}
	\end{frame}
	
	\begin{frame}{Semantics}
		the study of the meaning of natural language
		\begin{columns}
			\begin{column}{0.5\textwidth}
				\includegraphics[scale=0.25]{semantics.jpeg}
				
				\itshape{'I shot an elephant in my pyjamas.'}
			\end{column}
			
			\begin{column}{0.5\textwidth}
				\includegraphics[scale=0.25]{semantics2.jpeg}
				\includegraphics[scale=0.25]{semantics3.jpeg}
				\scriptsize{logic!}
			\end{column}
		\end{columns}
	\end{frame}
	
	\begin{frame}{Pragmatics}		
		\begin{columns}
			\begin{column}{0.5\textwidth}
				the study of the meaning of natural language in use
				
				\vspace{5mm}
				
				\includegraphics[scale=0.2]{maxims.jpeg}
				\itshape{'This traffic jam is really making my day!'}
			\end{column}
			
			\begin{column}{0.5\textwidth}
				\includegraphics[scale=0.4]{pragmatics.png}
				
				\vspace{5mm}
				
				\includegraphics[scale=0.15]{pragmatics2.jpeg}
			\end{column}
		\end{columns}
	\end{frame}
	
	\begin{frame}{Other fields}
		
		

			
	
		
		\begin{columns}
			\begin{column}{0.3\textwidth}
				
					\begin{itemize}
					\item Psycholinguistics
					\item Sociolinguistics
					\item Dialectometry
					\item Lexicography
					\item 
				\end{itemize}
			
			
			\end{column}
			\begin{column}{0.7\textwidth}
				\centering
				\includegraphics[scale=0.3]{psyling.png}
				\includegraphics[scale=0.3]{socling.png}
			\end{column}
		\end{columns}
		
		
		
	\end{frame}
	
\end{document}