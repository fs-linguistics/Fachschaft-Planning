\documentclass[aspectratio=169,hyperref={unicode}]{beamer}

\usetheme{Szeged}
\usecolortheme{beaver}
\usepackage{fontspec}
\usepackage{xcolor}
\usepackage{ulem}
\usepackage{hyperref}
\usepackage[german]{babel}


\title{Into to Gen Ling}
\author{John}
\institute{Fachschaft General \& Computational Linguistics\\ \textbf{University of Tübingen}}
\date{WS 2024/25 \\ Pre-course}

\begin{document}

\frame{\titlepage}


\begin{frame}{What is Linguistics}
  \begin{itemize}
\item The study of Language
          \item Scientific study of how language works, how they are structured, how it is used, and how it changes
    \item very broad field
          \item Some areas highly theoretical, some area very experimental
          \item Difference betweenn a linguist and a polyglot
  \end{itemize}
\end{frame}

\section{Progarm}

\begin{frame}{Program Overview}

  Basica
\end{frame}

\begin{frame}{Required classes}
\begin{itemize}
  \item Programming, Logic
        \item Phonology/Phonetics, Syntax, Semantic/Pragmatics
\end{itemize}

\end{frame}
\begin{frame}{Advanced Classes}
  \begin{itemize}
  \item Changes every semester
  \item List is shared with computational linguistics
  \item Format and workload depends on professor
  \item Examples intro to a subfield, such as Historical Linguistics.
  \item Intro to an uncommon language: Bororo
  \item A general overview of a topic: Language Evolution
  \end{itemize}
\end{frame}
\section{Career Opportunities}


\begin{frame}{Academia}
The lame answer
  \begin{itemize}
    \item Teaching and Research
	\item Potentially field work 
    \item \textbf{The only job that really requires a degree in linguistics}

  \end{itemize}
\end{frame}

\begin{frame}{Job Opportunities}
  \begin{itemize}
    \item Education: ESL, or other second language
    \item Translation
    \item Speach Pathologist
  \end{itemize}
\end{frame}


\begin{frame}{Cl Stuff}
But wait, I'm alergic to computers
\end{frame}

\begin{frame}{I am literally allergic to computers}
Some require a very basic level
  \begin{itemize}
    \item Annotation
    \item Data collection
    \item Data preprocessing
  \end{itemize}


\end{frame}

\end{document}

% Local Variables:
% TeX-engine: luatex
% End:
