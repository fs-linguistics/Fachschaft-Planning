% [[file:../roam/20230917220031-precourse_presentation.org::+begin_src latex :tangle ../precourse/presentation.tex :comments link][No heading:1]]
\documentclass[aspectratio=169,hyperref={unicode}]{beamer}

\usetheme{Szeged}
\usecolortheme{beaver}
\usepackage{fontspec}
\usepackage{xcolor}
\usepackage{ulem}
\usepackage{hyperref}
\usepackage[german]{babel}
\usepackage[utf8]{inputenc}
\usepackage{graphicx}
\usepackage{ulem}

\title{Into to Technology}
\author{John}
\institute{Fachschaft General \& Computational Linguistics\\ \textbf{University of Tübingen}}
\date{WS 2024/25 \\ Pre-course}

\begin{document}

% Title slide
\begin{frame}
\titlepage
\end{frame}

\section{Basic Commandline}

\begin{frame}{Outline}
\tableofcontents[currentsection]
\end{frame}



% Introduction to Command Line
\begin{frame}
\frametitle{What is a computer}

\begin{itemize}
\item Your computer a big calculator
        \item It has a list of instructions and it executes it one by one
\end{itemize}
\end{frame}

\begin{frame}
\frametitle{How to interact with the calculator}

\begin{itemize}
\item Usually you interact with a gui
        \item What did they do before guis?
\item They used the command line
\end{itemize}
\includegraphics[scale=0.5]{cmd.jpeg}
\end{frame}
% Navigating Directories
%

\begin{frame}[fragile]
\frametitle{What can you do with the Command Line?}
\begin{itemize}
        \item Short answer: Everything
        \item Simple examlpes: set a timer, do arithmetic, play a text based game
        \item Most importantly: run programs
\end{itemize}
\end{frame}


\begin{frame}[fragile]
\frametitle{The file system}
\begin{itemize}
  \item How does the computer know what to do?
        \item Files
        \item You should be familiar, when uploading photos
        \item A file is not just a photo, but any piece of data
        \item Example: Excel data, Word file, your Minecraft World
        \item Others: Settings, Programs, Folders, Mouse and Keyboard
\end{itemize}
\end{frame}

\begin{frame}[fragile]
\frametitle{Directories}
\begin{itemize}
\item Basically file manager but no mouse.
        \item Files are the exact same, just a different way to access them.
\item Example: \texttt{~/Desktop/homework/assignment1.java}

        \item Instead of clicking through folders you have to type in which folder you want to switch to.
        \item The names of folders separated by slashes is the path.
\end{itemize}
\end{frame}


% Viewing and Editing Files
\begin{frame}[fragile]
\frametitle{Other things you can do}
\begin{itemize}
        \item Many gui actions have command line equivalents
\item Opening a file it in notepad.
  \item Editing the file in Notepad
        \item Moving a file to the trash can
        \item Making a new file in Notepad
\end{itemize}
\end{frame}


% Introduction to Command Line for Python
\begin{frame}
\frametitle{But do I really need this?}
\begin{itemize}
        \item In our program, mostly for Text Technology, running python, and doing ssh
\item As a beginner, you can mostly get by by clicking the green button in VS Code
\item Sometimes green button no work, so its good to know how to do it from the command line
\item Make sure you do not develop command line phobia.
        \item No demo because too scary
\end{itemize}
\end{frame}



\section{Git}
% What is Git?

\begin{frame}{Outline}
\tableofcontents[currentsection]
\end{frame}

\begin{frame}
\frametitle{What is Git?}

\includegraphics[scale=0.3]{image.png}

\end{frame}

\begin{frame}
\frametitle{What is Git (for real)?}
\begin{itemize}
\item

\begin{quote}
Git  is a distributed version control system that tracks changes in any set of computer files, usually used for coordinating work among programmers who blah blah blah
\end{quote}
\item Basically google docs for code.
\item A command line program (but GUIs exist).
\item Most importantly, it's how you download code off of github and upload
\end{itemize}
\end{frame}

\begin{frame}
\frametitle{github}
\begin{itemize}
  \item git hub is the website, git is the actual program you use
        \item As an analogy, git is like Word, and Github is the Google Drive.
        \item Many classes require github
        \item Most open source softwre is hosted on github
        \item follow me on github btw
\end{itemize}
\end{frame}

\begin{frame}
But why can't I just use google drive/email/whats app/print out all my code on paper.
\end{frame}

\begin{frame}
\frametitle{Why use Git?}
\begin{itemize}
\item Keeps a history of all changes.
\item Easy collaboration through branching and merging capabilities. (meaning two people can work on different parts of the code base at the same time)
\item Distributed, meaning everyone that uses git has a copy of the source code, so you don't need internet to work on it.
        \item An industry standard that everyone expects you to know
        \item Can potentially show you did not cheat
\end{itemize}
\end{frame}

% Basic Commands
\begin{frame}
\frametitle{Basic Git Commands}
\begin{itemize}
  \item \texttt{git clone} Download your file, you only do this once.
  \item \texttt{git add} Selects which changes you want to upload
\item \texttt{git commit} Save your changes
\item \texttt{git push} Uploads you changes to github
\end{itemize}
\end{frame}


\begin{frame}
\frametitle{Workflow}
\begin{itemize}
  \item Do some work
        \item Save in your editor
        \item Git add your changes, (sometimes you don't want to add all your changes)
        \item Git commit, and give a message

        \item Git pull to see if anyone else made any changes. If there are changes you need to merge them.
        \item git push to see it on github.
        \item Repeat
\end{itemize}
\end{frame}

\begin{frame}
\frametitle{But that's too hard}
\begin{itemize}
  \item Download a program to help you
        \item github desktop, gitkraken,
        \item vscode also has git integration
        \item  Alternatively, edit directly on github (not recommended)

\end{itemize}
\end{frame}

\begin{frame}{Basically}
  Git can sometimes be hard
\includegraphics[scale=0.5]{git.png}
\end{frame}


\section{Debugging}
\begin{frame}{Outline}
\tableofcontents[currentsection]
\end{frame}

% What is Debugging?
\begin{frame}
\frametitle{But we didn't start writing code yet}
\begin{itemize}
  \item Debugging is an important skill
  \item More time is spent debugging than writing code
        \item Some debugging techniques are general that don't require use of a specfic tool
\end{itemize}
\end{frame}


% What is Debugging?
\begin{frame}
\frametitle{What is a bug?}
\begin{itemize}
        \item Basically, you are doing something wrong
\item It's a lot easier writing wrong code than correct code.
\end{itemize}
\end{frame}

% Types of Errors
\begin{frame}
\frametitle{Types of Errors}
\begin{itemize}
\item Compile Errors
\item Runtime Errors
\item Logic Errors
\end{itemize}
\end{frame}

% Compile Errors
\begin{frame}
\frametitle{Compile Errors}
\begin{itemize}
\item Basically, you made a grammar mistake
\item Computers are stubborn, they know what the error is but make you fix them
\item Examples: Missing semicolons, undeclared variables, mismatched brackets, typos
  \item Usually there will be a red squiggly line
        \item Relatively easy to fix, usually you copied something wrong.
\end{itemize}
\end{frame}

% Runtime Errors
\begin{frame}
\frametitle{Runtime Errors}
\begin{itemize}
\item Occur while the program is running.
  \item If compile error is a grammar mistake, a runtime error is a semantic mistake.
        \item Analogy, if you ask for the 10th person in line, but there are only 5 people in line.
\item Examples: Dividing by zero, trying to access an out-of-bounds array index, the famous null pointer exception
\item Usually your program crashes.
\item There is usually a line number to see where it failed
\end{itemize}
\end{frame}

% Logic Errors
\begin{frame}
\frametitle{Logic Errors}
\begin{itemize}
\item The code runs, but doesnt do what you want it to do
\item Examples: Incorrect formulas, missing steps in a process.
\item Often the hardest to detect because there are no explicit error messages.
        \item Issue is usally the programmar has made a mistake in their mental model.
        \item Analogy, you are baking, but the white powder you thought was sugar was salt.
\end{itemize}
\end{frame}

% Debugging Techniques
\begin{frame}
\frametitle{Debugging Techniques}
\begin{itemize}
\item Adding Print Statements
\item Using Paper and Pencil
\item Duck Debugging
        \item Using a debugger
\end{itemize}
\end{frame}

% Print Statements
\begin{frame}
\frametitle{Adding Print Statements}
\begin{itemize}
\item Adding lines of code to display variables or messages.
  \item Allows you to see which code is being run, which code is not.
    \item  Relatively simple, you learn hello world on the first day and that's all you need.
\end{itemize}
\end{frame}

% Paper and Pencil
\begin{frame}
\frametitle{Using Paper and Pencil}
\begin{itemize}
\item Write down variables and their values as you trace through the code.
\item Allows for manual simulation of how the code runs.
\item Sometimes writing stuff down just makes everything make sense
\end{itemize}
\end{frame}

% Duck Debugging
\begin{frame}
\frametitle{Duck Debugging}
\begin{itemize}
\item Explain your code or problem out loud, as if to a rubber duck or inanimate object.
        \item Sounds really stupid, but sometimes works
        \item if you do this in public, you may get bullied
\end{itemize}
\end{frame}



\begin{frame}
\frametitle{Debugger}
\begin{itemize}
        \item Will not go over, since it is technical
        \item Basically lets you go through your progarm line by line and obsveve variables
        \item Very powerful, learning how to effectively use is worth the effort
\end{itemize}
\end{frame}

\begin{frame}
\frametitle{Other useful tips}
\begin{itemize}
\item Google is your best friend
\item Test individual subcomponents
        \item Sometimes the most simple bugs are the most hard to find.
\item Go to sleep
\end{itemize}
\end{frame}
% Conclusion

\end{document}
%%% Local Variables:
%%% TeX-command-extra-options: "-shell-escape"
%%% TeX-engine: luatex
%%% End:
