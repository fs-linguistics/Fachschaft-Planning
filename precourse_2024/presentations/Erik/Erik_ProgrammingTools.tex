\documentclass[aspectratio=169,hyperref={unicode}]{beamer}
\usetheme{Szeged}
\usecolortheme{beaver}

\usepackage{fontspec}
\usepackage{xcolor}
\usepackage{ulem}
\usepackage{hyperref}
\usepackage[german]{babel}

\newcommand{\weblink}[2]{\textcolor{blue}{\uline{\href{#1}{#2}}}}

\title{Setting Up Programming Tools}
\author{Erik Zeiner}
\institute{Fachschaft General \& Computational Linguistics\\ \textbf{University of Tübingen}}
\date{WS 2023/24 \\ Precourse}
\begin{document}
\frame{\titlepage}
\begin{frame}{Load these slides to open links and copy commands}

	\centering \Large \url{https://fs-linguistics.github.io/precourse-schedule/}
	\vfill
	\large
	Make sure you download:
	\begin{itemize}
		\item ProgrammingTools.pdf
		\item PythonTest.py
		\item JavaTest.java
	\end{itemize}
\end{frame}

\begin{frame}{Installing Python}
There are a few options:
\begin{itemize}
	\item Pure Python [\url{https://www.python.org/downloads/}]
	\item Miniconda - good starter pack: python, few basic packages, tools to install further packages (conda, pip) [\url{https://docs.conda.io/projects/miniconda/en/latest/}]
	\item Anaconda - many, many tools useful for data science, including ML and NLP [\url{https://www.anaconda.com/download}]
\end{itemize}
There are different versions of Python, we will be installing version 3.11.
\end{frame}

\begin{frame}[fragile]{Installing Miniconda on macOS - Let's do a demo!}
\begin{center}
	\textbf{\url{https://docs.conda.io/projects/miniconda/en/latest/}}
\end{center}
\begin{verbatim}
mkdir -p ~/miniconda3
curl https://repo.anaconda.com/miniconda/Miniconda3-latest-MacOSX-arm64.sh 
-o ~/miniconda3/miniconda.sh
bash ~/miniconda3/miniconda.sh -b -u -p ~/miniconda3
rm -rf ~/miniconda3/miniconda.sh
\end{verbatim}
Environment: \begin{verbatim}
~/miniconda3/bin/conda init zsh
	conda activate
	conda deactivate
\end{verbatim}
\end{frame}

\begin{frame}[fragile]{Installing Miniconda on Linux - It's basically identical.}
\begin{center}
	\textbf{\url{https://docs.conda.io/projects/miniconda/en/latest/}}
\end{center}
\begin{verbatim}
mkdir -p ~/miniconda3
wget https://repo.anaconda.com/miniconda/Miniconda3-latest-Linux-x86_64.sh 
-O ~/miniconda3/miniconda.sh
bash ~/miniconda3/miniconda.sh -b -u -p ~/miniconda3
rm -rf ~/miniconda3/miniconda.sh
\end{verbatim}
Environment: \begin{verbatim}
~/miniconda3/bin/conda init zsh
	conda activate
	conda deactivate
\end{verbatim}
\end{frame}

\begin{frame}[fragile]{Installing Miniconda on Windows - Let's do a demo!}
\begin{center}
	\textbf{\url{https://docs.conda.io/projects/miniconda/en/latest/}}
\end{center}
\begin{verbatim}
curl 
https://repo.anaconda.com/miniconda/Miniconda3-latest-Windows-x86_64.exe 
-o miniconda.exe

start /wait "" miniconda.exe /S

del miniconda.exe
\end{verbatim}
Environment: Open app Anaconda Prompt
\end{frame}

\begin{frame}{Installing Java}
There are a few options:
\begin{itemize}
	\item Pure Java [\url{https://www.java.com/en/download/help/index_installing.html}]
	\item JDK (Java Development Kit) - software for working with Java applications\\ We will be installing Temurin by Adoptium [\url{https://adoptium.net}]
\end{itemize}
There are different versions of Java, we will be installing version 20. \\
"We need at least Version 11 of the JDK. Any newer version should also be fine."
\end{frame}

\begin{frame}[fragile]{Installing JDK on macOS/Linux - Let's do a demo!}
Firstly, let's install brew - a general package installer:
\begin{center}
	\textbf{\url{https://brew.sh}}
\end{center}

\begin{verbatim}
	/bin/bash -c "$(curl -fsSL 
	https://raw.githubusercontent.com/Homebrew/install/HEAD/install.sh)"
\end{verbatim}
Then, let's install the JDK using brew:
\begin{verbatim}
	brew install --cask temurin
\end{verbatim}
\textit {With Linux, you can use your preferred package manager.}
\end{frame}

\begin{frame}[fragile]{Installing JDK on Windows - Let's do a demo!}
\small
Install with an installer (.msi) from:\\

\url{https://adoptium.net/en-GB/temurin/releases/?os=windows&version=20&package=jdk}

\end{frame}

\begin{frame}{Tools for programmers}
\begin{itemize}
	\item Text editors
		\begin{itemize}
			\item Sublime Text
			\item Visual Studio Code
		\end{itemize}
	\item Integrated Development Environments (IDEs)
		\begin{itemize}
			\item IntelliJ IDEA - Java
			\item PyCharm - Python
			\item Spyder - Python
		\end{itemize}
	\item Notebooks
		\begin{itemize}
			\item Jupyter notebook 
			\item JupyterLab
			\item Google Collab
		\end{itemize}
\end{itemize}
	
\end{frame}

\begin{frame}{Visual Studio Code}
\begin{center}
	\url{https://code.visualstudio.com}
\end{center}
\begin{itemize}
	\item  General source-code editor 
	\item for Windows, Linux and macOS
	\item Support for debugging, syntax highlighting, intelligent code completion, snippets, code refactoring, and embedded Git. 
	\item Theme, keyboard shortcuts, preferences, and install extensions that add functionality.
\end{itemize}
\end{frame}

\begin{frame}[fragile]{Let's turn VS Code into a more proper IDE for Python!}
\begin{center}
	Download the extension: \\ \url{https://marketplace.visualstudio.com/items?itemName=ms-python.python}
\end{center}
	\begin{enumerate}
			\item Extensions settings
			\item Set \textit{Conda path} as your path to the miniconda3 folder
			\item Set \textit{Default Interpreter path} as your path to the miniconda3 folder
			\item Check if you can run the PythonTest.py file.
			\end{enumerate}
	\begin{verbatim}
		\Users\USERNAME\miniconda3 (macOS)
		
		C:/Users/USERNAME/miniconda3 (Windows)
	\end{verbatim}
\end{frame}


\begin{frame}{Let's turn VS Code into a more proper IDE for Java!}
		\begin{center}
		Download the extension: \\ \url{https://marketplace.visualstudio.com/items?itemName=vscjava.vscode-java-pack}	
		\end{center}
		\vfill
		\begin{enumerate}
			\item Check if you can run the JavaTest.java file.
			\end{enumerate}
\end{frame}

\begin{frame}{IntelliJ IDEA - Educational License}
\begin{center}
	\url{https://www.jetbrains.com/community/education}
\end{center}

\begin{center}
		For DSA II in the 2nd semester (or now if you prefer it)
\end{center}
\end{frame}
\begin{frame}
\begin{center}
\textbf{Make sure you have Python installed}

---

\textbf{Make sure you have Java installed}

---

\textbf{Make sure you have VS Code (or other IDE) installed}

\vfill
		If you have questions: \\ \href{mailto:erik.zeiner@student.uni-tuebingen.de}{erik.zeiner@student.uni-tuebingen.de}
		\\
		\url{https://chat.whatsapp.com/Djz1guUfBbx06rsRKQigfe}
\end{center}
\end{frame}
\end{document}