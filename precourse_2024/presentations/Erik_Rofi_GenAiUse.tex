\documentclass[aspectratio=169,hyperref={unicode}]{beamer}

\usetheme{Szeged}
\usecolortheme{beaver}
\usepackage{fontspec}
\usepackage{xcolor}
\usepackage{ulem}
\usepackage{hyperref}
\usepackage[german]{babel}


\title{Info on Generative AI Use}
\author{Rofaïda Rabehi \& Erik Zeiner}
\institute{Fachschaft General \& Computational Linguistics\\ \textbf{University of Tübingen}}
\date{WS 2024/25 \\ Pre-course}

\begin{document}


\frame{\titlepage}



\section{What is Generative AI?}

\begin{frame}
	basic terminology
\end{frame}

\begin{frame}
	how does an llm work
\end{frame}

\section{What is it good for?}

\begin{frame}
	rofi
\end{frame}

\section{What are the limitations?}

\begin{frame}
	theoretical limitations
\end{frame}

\begin{frame}
	example of limitations translation
\end{frame}

\begin{frame}
	example of limitations - cl related
\end{frame}

\section{Should you (not) use it for university work?}

\begin{frame}
	rofi
\end{frame}


\begin{frame}
\begin{center}

\textbf{Think critically, like a computational linguist}

---

\textbf{Don't rely on it}

---

\textbf{Use with care and consideration}


\vspace{1em}

\begin{minipage}{0.4\textwidth}
\centering
    \includegraphics[width=0.5\textwidth]{QRtemplate_5.png}
  \end{minipage}
  \hfill
  \begin{minipage}{0.4\textwidth}
  \centering
    \includegraphics[width=0.5\textwidth]{QRtemplate_4.png}
  \end{minipage}
\end{center}
\end{frame}
\end{document}